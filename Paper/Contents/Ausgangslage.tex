\section{Ausgangslage}
\label{sec:ausgangslage}
% Dennis

% Weniger KI, mehr SWE II
% Du beschreibst hier schon die Systemarchitektur

% Erst alles historische (und hysterische) zu Genderly
% Dann "Brücke" zu Weiterentwicklung Equaly
% Dann sagen, warum Weiterentwicklung (weil eben so in Semester 4 erwartet und weil Verbesserungsvorschläge erarbeitet wurden)

Equaly ist eine Weiterführung des Softwareprojekts Genderly des Vorgängermoduls ''Software Engineering I'' aus dem dritten Semester. Genderly dient dadurch als Grundlage für die Weiterentwicklung unter dem neuen Namen Equaly.

Genderly hatte das gleiche allgemeine Ziel wie das jetzige Equaly, jedoch wurde in der Entwicklung ein anderer Ansatz verfolgt als in Equaly. Das Ziel von dem Modul ''Software Engineering I'' war, einen funktionsfähigen Prototyp zu erarbeiten. Im Vorfeld ist es jedoch nötig gewesen, sich in die Thematik der Verarbeitung natürlicher Sprachen einzuarbeiten. Das Ergebnis dieser Arbeit war eine auf Java basierende Webanwendung mit HTML5-Frontend und rudimentärer Datenbank. In deutschen Sätzen wurden Worte mit Großbuchstaben herausgefiltert und gegen die Einträge dieser Datenbank geprüft. Sofern hier ein genderneutrales Ersatzwort gefunden wurde, wurden Vorgänger und Nachfolger des Wortes darauf geprüft, ob sie Artikel sind. War das der Fall, wurden auch sie an das neue Substantiv angepasst.

Die Anforderung des Nachfolgemoduls ''Software Engineering II'' bestand nun daraus, die Weiterentwicklung dieses Softwareprojektes zu einer ausgereifteren Softwarelösung durchzuführen.