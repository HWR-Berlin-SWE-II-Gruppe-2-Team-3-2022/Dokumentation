\section{Zielgruppenanalyse}
\label{sec:zielgruppenanalyse}
% Felix

Bei der Entwicklung von Applikationen muss darauf geachtet werden, für welche Zielgruppen die Anwendung entwickelt wird. Die Zielgruppen sollen die zukünftigen Anwender*innen der Software sein. Sie bilden einen wesentlichen Teil der Stakeholder*innen.

Da im akademischen Raum das Gendern umfassend an Bedeutung gewinnt, ist entsprechend davon auszugehen, dass es sich bei den zukünftigen Anwender*innen zu einem großen Anteil um Studierende handeln wird. Deshalb orientiert sich die Anwendung insbesondere an dieser speziellen Zielgruppe. Die Applikation soll die Studierenden beim einfachen und korrekten Gendern in wissenschaftlichen Texten unterstützen. Für Student*innen ist es wichtig, dass sie einen geringen Arbeitsaufwand bei der Nutzung der Software erfahren. Insofern muss die Anwendung einfach benutzbar sein.

Die Software kann auch von Dozent*innen genutzt werden. Die Verwendung der Applikation findet in diesem Fall vorwiegend im didaktischen und pädagogischen Zusammenhang statt. Die Dozent*innen beabsichtigen dabei, ihren Lehrauftrag zu erfüllen, welcher indirekt auch die Vermittlung der Gleichberechtigung der Geschlechter beinhaltet. Dafür muss die Software einfache, flexible und mobile Anwendungsmöglichkeiten bereitstellen.

Eine weitere Zielgruppe sind Lehrkräfte \& Schüler*innen. Die Lehrkräfte haben wie die Dozent*innen ein ausgeprägtes Interesse an der einfachen und flexiblen Anwendung des Produkts. Sie verfolgen ebenso das Ziel, ihren Lehrauftrag zu erfüllen, welcher die Vermittlung der Gleichberechtigung der Geschlechter beinhaltet. Schüler*innen möchten gleichzeitig die eigentlichen Textinhalte sehr leicht lernen können. Beide Gruppen nutzen die Software als Hilfsmittel während des Schreibens. Besonders zu betonen ist hierbei das potenzielle Alter der Anwender*innen. Sowohl sehr junge, als auch sehr erfahrene Menschen sollen die Software nutzen können. Equaly soll die Nutzer*innen bei der Erweiterung ihres Wissens über genderneutrales Schreiben unterstützen. Wegen der großen Altersspanne ist eine stark vereinfachte Bedienung erforderlich.

Weitere mögliche Anwender*innen der Software sind Politiker*innen und Journalist*innen. In Teilen der Politik und des Journalismus wird vermehrt Wert auf inklusives Schreiben gelegt. Das Ziel dabei ist es, dass sich dass Publikum besser angesprochen und stärker einbezogen fühlt, damit die Politiker*innen oder Journalist*innen eine möglichst große Wählerschaft beziehungsweise eine möglichst große Leserschaft erreichen. Diese Zielgruppe hat jedoch keinen großen Einfluss auf die Anforderungen der Applikation, deswegen wird sie in diesem Absatz nur kurz erwähnt, im folgenden Text aber nicht weiter auf sie eingegangen.