\section{Fazit}
\label{sec:fazit}
%Marcus

Mit der vorliegenden Arbeit beschreiben wir die Anwendung der Phasen, Techniken und Strategien der Softwareentwicklung anhand eines praktischen Kursprojekts. Die Erarbeitung und Entwicklung fand im Rahmen des Moduls SWE II der HWR Berlin statt. Im Rahmen des Kurses wurde die Entwicklung einer Software zur Adressierung eines der 17 Ziele der United Nations vorgegeben. Das Team entschied sich, Software als Beitrag für die Geschlechtergleichheit zu entwickeln \cite{Un21}. Dazu wurde eine Zielgruppenanalyse durchgeführt. Die erkannten Zielgruppen wurden u.A. mit Gewohnheiten, Arbeitsumfeld und Art der Arbeit als Grundlage für die Architektur der Software, das Design der Oberfläche und die Art der Bereitstellung genutzt. Für diesen gesamten Prozess, sowie für die resultierende Implementierungsarbeit, wird die Arbeitsweise beschrieben. Zentral werden ebenfalls die Maßnahmen der Qualitätssicherung, wie auch ein durchgeführter Beta-Test beleuchtet.

Ergebnis ist die in \ref{fig:equaly} gezeigte Web-Applikation Equaly. Als Ergebnis eines weiterentwickelten Prototyps kann Equaly unter Verwendung von Java und den Frameworks Spring, OpenNLP und Lingua eingegebenen Text aufnehmen, zerlegen, analysieren und zu einem genderneutralen Textvorschlag umformen. Diese Funktionalität wurde für die deutsche und die englische Sprache realisiert. Die Software erfüllt die erwartete Funktionalität. Bestimmte Ungenauigkeiten in der Vorschlagserzeugung mussten zeitbedingt erhalten bleiben. Diese Problemstellungen versprechen eine interessante Grundlage für zukünftige Arbeiten mit NLP-Technologien hin zu einer Anwendung für genderneutrale Textgenerierung zu sein.