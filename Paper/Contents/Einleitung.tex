\section{Einleitung}
\label{sec:einleitung}
% Dennis

% Erklären Sie hier gerne, warum Sie das Projekt machen, warum es wichtig ist und wo die Motivation her kommt. Eine Forschungsfrage könnte sich unter anderem daraus ergeben, wenn Sie fragen, wie Sie das Projekt umsetzen können. Also sowas ähnliches wie "Wie kann das Projekt XYZ unter Berücksichtigung des Softwarelebenszykluses erfolgreich implementiert werden?"


%Die vorliegende wissenschaftliche Ausarbeitung beschreibt den Entwurfs- und Entwicklungsprozess der Assistenzwebanwendung "Equaly", als des Moduls „Software Engineering II“.

%Dabei handelt es sich um die Weiterführung des Softwareprojekts „Genderly“ des vorhergehenden Moduls „Software Engineering I“.\\
%„Genderly“ sowohl als auch „Equaly“ dienen dem Nutzer als Hilfe bei der Verfassung von Texten, indem sie dem Nutzer das Einhalten von Gendernormen erleichtern.
%Um Dies zu erreichen, wird der Text analysiert und anschließend wird dem Nutzer ein bearbeiteter Text, in Form eines Gendervorschlages, präsentiert. 
%Der generierte Gendervorschlag basiert auch verschiedenen Faktoren, unteranderem auch dem Genderstil. Während der Entwicklung wurden verschiedene Genderstile festgelegt und sogenannte „Nice to Haves“ und „Must Haves“ aufgeteilt. Während Aufteilung dieser „Nice to Haves“ und „Must Haves“ wurde entschieden, dass der primäre Genderstil der „Wortersatz“ sein wird.
%Diese Fähigkeit des Wortersatzes wurde zur Kernfunktion unserer Anwendung und mit Hilfe einer Datenbank und KI Pipeline realisiert.
%Die Entwicklungs- und Dokumentationsphasen, die zur Finalisierung des Softwareprojektes „Equaly“ geführt haben, werden in der folgenden Ausarbeitung thematisiert und dargestellt.  

% Wie wird an der HWR Berlin Software Engineering vermittelt.
% (Paper von Frau Monett Díaz zitieren, wenn über Struktur und Aufbau des Unterrichts kurz geredet wird)
% "im 3. Semester das und das gemacht, im 4. Semester das und das ..."
% "Im Rahmen dieses Moduls ... Entwicklungsprinzipien lernen und praktisch anwenden"
% "Aufgabe: Eine Software für eines der 17 Ziele der UN entwickeln, durch alle Phasen des Softwareengineerings führen"
% "Dabei auch Qualitätssicherung und Einholen von Feedback praktiziert"


Das Verfassen von Texten verschiedenster Art, unabhängig von ihrem Inhalt, hat immer das Ziel, möglichst viele Leser*innen der gewählten oder auch weiterer Zielgruppen anzusprechen. Dies kann auf verschiedenste Art und Weise erreicht werden, z.B. durch bestimmte Stilmittel. Das Einbinden aller Geschlechter (Gender) in die Zielgruppenfindung und die Sprache stellt dabei einen der effektivsten Wege dar, eine breite Leserschaft zu adressieren.

Um dies zuverlässig zu erreichen, wird von den Autor*innen erwartet, ihre Texte in grundlegenden Aspekten an die Erwartungen der Leserschaft anzupassen. Der Anpassungsprozess durch z.B. Umformulierungen kann dabei viel Zeit und Energie der Autor*innen in Anspruch nehmen. Diese Änderungen und ihr Aufwand treten nicht nur während des Verfassens, sondern u.U. auch nach der inhaltlichen Fertigstellung eines Werkes auf. Das kann der Fall sein, wenn sich die von der Gesellschaft und der gewünschten Leserschaft erwarteten Gendernormen im Nachhinein verändern.

Genau an dieser Stelle besteht der Bedarf für Equaly. Mit Hilfe von automatischer Texterkennung und Textanalyse ist es in der Lage, genderspezifische Ausdrücke mit genderneutralen Begriffen auszutauschen. 

Equaly ist das Ergebnis einer Gruppenarbeit im Rahmen der Module  ''Software Engineering I'' und ''Software Engineering II'' des Informatik Fachbereichs der Hochschule für Wirtschaft und Recht Berlin.
Diese Module fanden während des 3. und 4. Semesters des Studiums statt. Die Module hatten das Ziel, im Laufe eines Semesters und begleitend zu den Vorlesungen und Laboren den Studierenden zu ermöglichen, ihr eigenes Softwareprojekt zu planen, zu dokumentieren und eine Softwarelösung zu entwickeln.
Dabei wurde gelehrt, diverse Entwicklungsprinzipien praktisch anzuwenden.

Die im Modul gegebene Aufgabenstellung für das Softwareprojekt war es, eine Software für eines der 17 Ziele der UN zu entwickeln. Dabei galt es, alle Phasen des Software-Engineerings zu durchlaufen. Innerhalb des Entwicklungsprozesses wurden so auch Qualitätssicherungspraktiken durchgeführt und das Feedback von Testpersonen im Rahmen eines Beta-Tests eingeholt.