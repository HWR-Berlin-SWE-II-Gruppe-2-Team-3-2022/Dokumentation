\section{Softwarearchitektur}
\label{sec:softwarearchitektur}
% Marcus

Basierend auf der erstellten Zielgruppenanalyse gilt es, die Architektur der zu entwickelnden Software zu entwerfen. Diese stellt für das System die fundamentale Organisation durch Komponenten, ihre Beziehungen und ihre Umwelt dar \cite{Hi00}. Die resultierende Architekturbeschreibung dient zusätzlich als Dokumentation und als wesentliche Kommunikationsgrundlage für die Stakeholder*innen des Projekts.

Mit dem strategischen Entwurf der Architektur wird insbesondere berücksichtigt, dass Vertreter der ausgemachten Zielgruppen die Software optimal erreichen, nutzen und unkompliziert von ihrer Funktionalität profitieren können. Bei den ermittelten Zielgruppen ist jeweils eine hohe Variation verwendeter Endgeräte festzustellen. Mit unterschiedlichen Endgeräten werden wiederum sehr unterschiedliche Arbeitsaufgaben wahrgenommen. Das Verfassen journalistischer Artikel kann beispielsweise entsprechend in einer statischen Büroumgebung, aber auch mittels mobiler Endgeräte stattfinden. Eine ähnliche Vielfalt kann im akademischen Bereich für Schüler*innen, Dozent*innen und Student*innen bei der Erarbeitung von Texten, Berichten und wissenschaftlichen Ausfertigungen beobachtet werden.

Die Softwarearchitektur ist entsprechend derartig zu gestalten, dass eine schließlich realisierte Softwarelösung für eine größtmögliche Anzahl potenzieller Nutzer*innen verwendbar ist. Gleichzeitig soll sie für eine Vielzahl von Endgeräten einheitliche Funktionalität bieten, da andernfalls der Entwicklungsaufwand das Zeitbudget des Kurses übersteigen würde. Basierend darauf wurde entschieden, Equaly als Web-Applikation zu entwickeln. Unterschiedliche Geräte können dadurch mit niedrigem lokalen Ressourcenbedarf Ergebnisse anzeigen, da die Geschäftslogik mit dieser Art der Software auf einen Webserver ausgelagert werden kann. Die Faktoren der Zuverlässigkeit und der Anpassbarkeit von Equaly werden dadurch zentral für das Entwicklerteam adressierbar. Der Zugriff auf Web-Applikationen erfolgt parallel durch mehrere Nutzer mittels Webbrowser, welcher statt einer eigenen, spezialisierten Client-Software die Rolle des Thin-Client einnimmt \cite{Wu01}.

Mit Festlegung der Anwendungsart muss die Frage nach den zu verwendenden Technologien beantwortet werden.
Aufgrund von Vorwissen und Erfahrung innerhalb des Teams wird für das Backend der Web-Applikation Java unter Verwendeung des Spring-Frameworks genutzt. Für das Frontend werden HTML, CSS, JavaScript und Thymeleaf verwendet. Letzteres dient als Übertragung der Informationen zwischen Frontend und Backend.

Die geplante Kernkompetenz der Anwendung, die Sprachverarbeitung, muss beim Aufbau der Architektur in seiner Komplexität berücksichtigt werden. Mit Feedback zum bereits eingereichten Prototyp konnte festgestellt werden, dass für die Sprachverarbeitung ein deutlicher Bedarf für den Einsatz von künstlicher Intelligenz besteht. Aufgrund von vorhergehenden Erfahrungen im Team findet das vorab trainierte KI-Modell der Bibliothek Lingua für die Sprachermittlung Verwendung. Mehrere ebenfalls vortrainierte KI-Modelle aus der Bibliothek OpenNLP realisieren dann Textanalysen und Klassifikationen. Zur geplanten Realisierung eines Dictionaries wird eine SQLite-Datenbank genutzt. Hier sollen genderbare Wörter und ihre jeweiligen Alternativen eingetragen und verfügbar sein, wie auch in \ref{fig:c2} mit der C2-Architekturdarstellung gezeigt ist.
Als Architekturmuster wird, aufgrund der bis hierhin klar unterteilbaren Softwarekomponenten, das Architekturmuster Model-View-Controller (MVC) genutzt.