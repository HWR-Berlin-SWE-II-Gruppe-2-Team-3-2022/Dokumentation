\section{Arbeitsweise}
\label{sec:arbeitsweise}
% Felix

Equaly ist ein Softwareprojekt und benötigt zwecks seiner Weiterentwicklung eine Projektorganisation. Dadurch wird die Arbeit innerhalb des Projektteams strukturiert und im Ablauf optimiert. Beim vorliegend betrachteten Softwareprojekt ist das gesamte Team für die Organisation verantwortlich. Das Team arbeitet grundsätzlich zusammen. Bei den zu erledigenden Aufgaben bringt sich jedes Teammitglied mit ein. Für die Kommunikation innerhalb des Teams wurde eine Discord-Gruppe erstellt. Das Team kommuniziert hier regelmäßig zu wöchentlichen Standups, diskutiert über die noch zu erledigenden Aufgaben und plant hier das weitere Vorgehen.

Des Weiteren wurde für das Softwareprojekt eine GitHub-Organisation\footnote{Link zur GitHub-Organisation: \href{https://github.com/HWR-Berlin-SWE-II-Gruppe-2-Team-3-2022}} mit je einem Dokumentations- und einem Software Repository gegründet. Durch die Nutzung von GitHub kann das Team gemeinsam an der Software arbeiten, den geschriebenen Code strukturiert verwalten und überprüfen. Hinzu kommt, dass eine automatische Versionierung mittels Git von Beginn an sichergestellt wird. Die Repositories sind jeweils öffentlich einsehbar, um den Entwicklungsprozess im Rahmen des Moduls nachvollziehbar und transparent zu gestalten.

Eine klar abgegrenzte Rollenverteilung ist bewusst nicht vorhanden. Jedoch hat jedes Teammitglied ein Spezialgebiet, sodass je nach Aufgabe ein Mitglied das Grundgerüst erstellt, welches vom Rest des Teams durch Anmerkungen und Reviews ausgearbeitet und verfeinert wird. Dennis ist so für das Testing, die Test Infrastruktur und das Mocking zuständig. Felix erarbeitet in Abstimmung die Dokumentation, das Wording und das Marketing. Marcus ist für die Entwicklung zuständig.