\section{Bereitstellung}
\label{sec:bereitstellung}
%Felix

In der Zielgruppenanalyse wurden mehrere Anforderungen herausgearbeitet, welche die Anwendung erfüllen muss, damit eine hohe Nutzungsrate erreicht wird. Eine einfache Benutzbarkeit, sowie eine einfache, flexible und mobile Verwendung der Applikation sind von Relevanz. Die einfache Benutzbarkeit wird durch eine schlichte und intuitive Bedienoberfläche sichergestellt, während für die flexible und mobile Verwendung die Art der Bereitstellung der Software eine Rolle spielt.

Es wurde beschlossen, dass Equaly als Webanwendung ausgelegt wird. Da ein Browser mit Internetzugriff auf fast jedem netzwerkfähigem Gerät vorhanden ist, können die Nutzer*innen das Tool plattformunabhängig verwenden. Die Anwendung der Software ist dadurch nicht mehr auf ein bestimmtes Betriebssystem oder eine Gerätart beschränkt.

Des Weiteren ist auch der Betrieb des Webservers für die Bereitstellung der Anwendung plattformunabhängig möglich. Das Backend wurde mittels Java 8 implementiert. Java ist für alle gängigen Betriebssysteme verfügbar, sodass auf jedem Rechner, auf dem Java installiert ist, der Webserver für Equaly eingerichtet werden kann. Zusätzliche Frameworks, Bibliotheken und Techniken, die bei der Implementation des Backends zum Einsatz kamen, sind alle innerhalb einer JAR-Datei integriert, sodass die einzige Voraussetzung für die Einrichtung des Servers die Installation von Java 8 ist. \\
Der Quellcode und eine Installationsanleitung werden veröffentlicht, damit Institutionen und Individuen mit Interesse an dem Softwareprojekt sich es runterladen und ohne Probleme ausführen können.

% Bei der Bereitstellung kannst du zusammenfassend kurz darauf eingehen, welche Zielgruppen wir identifiziert haben und wie die bestmöglich erreicht werden können: Mit der entwickelten Web-Anwendung (das sollten maximal 2-3 Sätze sein, weil diese Entscheidung hier ja schon getroffen (und oben schon angesprochen) wurde)
% Du konkretisierst dass dann aber z.B. dahingehend, das die Verwendung von Java uns den Faktor der Plattformunabhängigkeit für den Betrieb des Backends erlaubt hat.
% Dann kannste auch sagen, dass die Kombination aus den verwendeten Frameworks, Bibliotheken und Techniken eine Integration aller Notwendigkeiten in eine einzige JAR-Datei ermöglichte
% Als einzige Voraussetzung braucht man also nur Java 8
% Der Quellcode und eine Installationsanleitung werden veröffentlicht, der Betrieb ist zunächst so gedacht, das Institutionen mit Interesse an diesem Projekt dieses runterladen und dann eben easy laufen lassen können