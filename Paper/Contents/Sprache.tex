\section{Genderneutrale Sprache}
\label{sec:sprache}
% Dennis

Sprache ist allgegenwärtig und dient nicht immer nur als Kommunikationsmittel. 
Sie kann informieren, aber auch unterhalten. Parallel dazu wandelt sie sich so beständig, wie die Gesellschaft. Dies ist unter anderem auch im Gebiet des Genderns zu beobachten. Gendern wird zusehens Teil der alltäglichen Sprachpraxis. Gendergerechte Sprache ist ein weiterer Schritt im immer andauernden Entwicklungsprozess der Sprache und Sprachpraxis. Sie stellt eine Möglichkeit dar, Informationen durch eine diversere Sprachvielfalt inklusiver zu gestalten und stellt damit einen Beitrag zur Gleichberechtigung dar.

Diese ursprüngliche, dominant männliche Sprachorientierung wird mit Hilfe des generischen Maskulinums realisiert. Wenn das generische Maskulinum genutzt wird, werden allgemeine Personengruppen mit der männlichen Form dargestellt. Dabei wird zum Beispiel eine Gruppe aus Studierenden als ``Studenten'' bezeichnet. Ohne dabei zu berücksichtigen, dass sich in der Gruppe möglicherweise weibliche Studenten -sprich Studentinnen- befinden können. Dies ist eine weit verbreitete, da historisch gewachsene sprachliche Konvention \cite{Hei00}.

Diversifikation der Sprache findet sowohl verbal, als auch schriftlich statt. In textueller Form wird dies durch die richtigen Sprachmittel erreicht. Eine Möglichkeit, Genderneutralität in einem Text zu erreichen, ist die genderneutrale Sprache durch Einhaltung von Gendernormen. Diese Gendernormen können wiederum unterschiedliche Regelsätze zur Wortbildung bzw. Umformulierung bedeuten.

Einer dieser Regelsätze wird zum Beispiel mit der Verwendung des Gendersterns angewendet. Bei dieser Schreibweise wird versucht, das männliche und das weibliche Geschlecht in gemeinsamen Ausdrücken zu vereinen. Beispielsweise wird dadurch eine Gruppe von Studierenden nicht mehr als Studenten, sondern als ``Student*innen'' bezeichnet. Der Genderstern agiert hierbei als Verbindungssymbol zwischen dem generischen Maskulinum und der Wortendung ``-in/-innen''. Alternativ lässt sich der Genderstern auch durch andere Symbole, wie zum Beispiel den Doppelpunkt, das Semikolon oder auch den Schrägstrich ersetzen. Diese Alternativen bilden Beispiele von Genderstilen der Schrift. Ein Genderstil, der bei einer Untermenge genderbarer Wörter angewendet werden kann, ist der vollständige Wortersatz. Dadurch kann eine bessere Lesbarkeit eines Textes gewährleistet werden. Jedoch fordert dieser Genderstil die meiste Aufmerksamkeit des Autors.

Bei der Nutzung des Genderstils des vollständigen Wortersatzes werden genderspezifische Begriffe durch genderneutrale Alternativbegriffe ersetzt. So wird beispielsweise eine Gruppe von ``Männern'' als eine Gruppe von ``Personen'' bezeichnet. Der Vorteil dieses Genderstils ist, dass der Lesefluss eines Textes, anders als bei der Nutzung des Gendersterns, nicht beeinträchtigt wird. Die Beeinflussung des Leseflusses wird beim lauten Vorlesen eines Textes besonders deutlich, da das Aussprechen des Gendersterns beispielsweise keine konkrete Vorgabe in der deutschen Sprache hat. Ein weiterer Nachteil des Gendersterns, den der Wortersatz nicht besitzt, ist, dass einige Wörter nicht gegendert werden können. Beispielsweise kann die Bezeichnung ``Mann'' nicht zu ``Mann*in'' umgeformt werden. Mit Hilfe des Stils des Wortersatzes ist Gendering hier hingegen möglich.