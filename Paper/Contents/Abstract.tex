\begin{abstract}
\boldmath
Der geschlechterbewusste und inklusive Sprachgebrauch findet insbesondere im pädagogischen, wie auch im akademischen und medialen Bereich immer mehr Verwendung. So werden unter anderem wissenschaftliche Arbeiten von Student*innen sowie journalistische Artikel und Recherchen in genderneutraler Sprache verfasst. Das Schreiben dieser genderneutralen Texte kann eine zusätzliche Herausforderung für die Autor*innen bedeuten. Unterschiedliche Schreibstile bedürfen individueller Eingewöhnungszeiten. Um dieses Problem anzugehen, schlagen wir im vorliegenden Paper im Rahmen des Kursprojekts des Moduls Software Engineering II (SWE-II) der HWR Berlin eine neuartige Softwarelösung vor, welche Text in eine genderneutrale Form bringt. Unser Ansatz zielt darauf ab, eine zielgruppengerechte, durch Endanwender*innen nutzbare Möglichkeit für das automatisierte Gendern von Texten zu bieten. Basierend auf gängigen Regeln des Genderns in deutscher und in englischer Sprache werden hierfür die technischen Möglichkeiten des \textbf{Natural Language Processing (NLP)} betrachtet. Die zugehörigen Phasen der Softwareentwicklung und die Arbeitsweise mit der Planung, der Erarbeitung der Softwarearchitektur, der Implementierung und der Qualitätssicherung werden im Kontext des Moduls SWE-II beschrieben.
\end{abstract}