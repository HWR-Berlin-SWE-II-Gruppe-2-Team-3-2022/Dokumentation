\section{Diskussion}
\label{sec:diskussion}

Im Rahmen des Moduls ``Software Engineering II'' wurden unterschiedliche Themen behandelt, um den Student*innen die notwendige Theorie näherzubringen, die benötigt wird, um das jeweilige Softwareprojekt unter Durchlauf der Phasen des Software-Engineering realisieren zu können.

Dabei wurden die Vorlesungen des Moduls unter drei unterschiedlichen Professor*innen und Dozent*innen aufgeteilt, wobei jede Person einen anderen Aspekt der Thematik übernommen hat. Anhand der Projektarbeit konnten die Studierenden praktische Erfahrung sammeln, wie der Entwicklungsprozess einer Software aufgebaut ist. Es wurden unter anderem Grundsätze, Entscheidungsschemata und Methodiken für den Entwurf von Bedienoberflächen gelehrt. Außerdem wurde sich innerhalb des Kurses umfassend mit der Implementierung, der Qualitätssicherung und dem Testen, sowie der Dokumentation und der Auslieferung von Software befasst.

Von Anfang an mussten die Student*innen strukturiert an das Projekt herangehen und sich über das gesamte Semester hinweg wie professionelle Entwicklerteams selbstständig organisieren. Diese Selbstständigkeit gepaart mit der Disziplin innerhalb des Projektteams, dem gesammelten Wissen und der gesammelten Erfahrung wird in Zukunft bei möglichen anderen Projekten sehr hilfreich sein.
% Was schreib ich hier?
Anzumerken ist, dass wir (leider) zu Beginn des Projekts die Relevanz unserer Software unterschätzt haben. Equaly hat das Potenzial ein Tool des alltäglichen Gebrauchs zu sein, welches sowohl von Personen der Öffentlichkeit, als auch von Privatpersonen benötigt wird. Falls wir uns dazu entschließen, ein weiteres Softwareprojekt mit einem ähnlichen Ziel zu starten, werden wir in Betracht ziehen, nach offizieller Unterstützung von Sponsoren zu suchen.
% Seriously, was schreib ich hier???